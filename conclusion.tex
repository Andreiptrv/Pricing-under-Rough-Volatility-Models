\section{Conclusion}
    We got aquainted with the fractional stochastic volatility models framework and
    studied the statistical properties of RFSV. 
    We obtained roughness estimations for major Russian companies stocks and depositary 
    reciepts, and reproduced some effects described in \cite{GatheralRosenbaum2014}.

    \paragraph*{Reproduced hypotheses:}
    \begin{enumerate}
        \item The Hurst exponent of the considered assets has the order of $1e-1$ and is less than $\frac{1}{2}$.
        \item The volatility of the considered assets \textbf{does not} have a property of long memory under fractional stochastic 
                volatiltiy models.
        \item Visual analysis and normality tests for the log-increments of volatility shows that for 
              $\Delta \in [10, 25]$ the normality of log-inrements hypothesis holds.
        \item The smoothing effect holds for the estimations of $H$ and $\alpha$ (volatility of volatility under fOU). 
              But \textbf{only} for  VTBR LI Equity we got a negative slope of the smoothing effect. For other
              asset we got a nearly perfect linear fit and positive smoothing slopes.
    \end{enumerate}