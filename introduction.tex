The first revolutionary model of Mathematical Finance was introduced by F. Black and
M. Sholes in 1973's article \cite{BlackSholes1973}, and a similar model introduced in
1976 by F. Black in \cite{Black1976} (the only fundamental difference was the use of 
the forward prices instead of spot prices, which is proven to be useful for some markets). 
Later there were created some local volatility models (like Dupire's LVM), and first 
stochastic volatility models (Heston SVM), but they still were not a perfect fit for pricing, 
even when first LSVMs were introduced.

Fractional Brownian motions were first employed in volatility modelling by F. Comte and 
E. Renault \cite{ComteRenault1998}. Their model (called FSV) used a fractional Brownian 
motion with Hurst parameter $H > 0.5$ to model volatility as a long-memory process i.e. 
one where autocorrelation decays slowly, which was a widely accepted stylized fact. They 
thus introduced the class of fractional stochastic volatility models.

In 2014, J. Gatheral, T. Jaisson, and M. Rosenbaum showed in \cite{GatheralRosenbaum2014} 
that for major American indices Hurst parameter estimations have the order of $1e-1$ 
(i.e. $H < 0.5$, therefore, there is no long-memory in the FSV-based models), and called 
the corresponding model (FSV + $H < 0.5$) a rough fractional stochastic volatility model 
(RFSV) to emphasise that the volatility is indeed rough.

However, their approach requires the use of a model, therefore, it is not perfect still.
In 2022, R. Cont and P. Das \cite{Cont2022} proposed a method of estimating the roughness of 
an asset without the need of a model, which can be used to find statistical evidence that
volatility is rough even without RFSV.

In the present paper I show that the Hurst parameters of the major Russia-originated 
assets (stocks and depositary reciepts of Russian corporations) are less than $0.5$ under RFSV, 
i.e. Comte and Renault's basic FSV model is not working well for the Russian stock markets, 
therefore, RFSV should be used instead.