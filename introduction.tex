One of the most famous models of mathematical finance was introduced by F. Black and
M. Sholes in 1973's article \cite{BlackSholes1973}, and a similar model for forward prices 
introduced in 1976 by F. Black in \cite{Black1976}. 
Later there were invented some local volatility models, and stochastic volatility models 
(Heston, Hull and White, SVI, SABR etc.), but they still were not a perfect fit for pricing, 
even when first LSVMs were introduced.

Fractional Brownian motions were employed in volatility modelling by F. Comte and 
E. Renault in \cite{ComteRenault1998}. Their model (called FSV) used a fractional Brownian 
motion with Hurst parameter $H > 0.5$ to model volatility as a long-memory process i.e. 
one where autocorrelation decays slowly, which used to be a widely accepted stylized fact. They 
thus introduced the class of fractional stochastic volatility models.

In 2014, J. Gatheral, T. Jaisson, and M. Rosenbaum showed in \cite{GatheralRosenbaum2014} 
that for major American indices Hurst parameter estimations are consistently less than $0.5$, 
They called the corresponding model (FSV, $H < 0.5$) a rough fractional stochastic volatility model 
(RFSV) to emphasise that the volatility is indeed rough.

However, their approach requires the use of a model, therefore, it is not perfect still.
In 2022, R. Cont and P. Das \cite{Cont2022} proposed a method of estimating the roughness of 
an asset without the need of a model, which can be used to find statistical evidence that
volatility is rough even without RFSV.

In the present paper we show that the Hurst parameters of the major Russia-originated 
assets (stocks and depositary reciepts of Russian corporations) are less than $0.5$ under RFSV, 
i.e. Comte and Renault's basic FSV model is not working well for the Russian stock markets, 
therefore, RFSV should be used instead.