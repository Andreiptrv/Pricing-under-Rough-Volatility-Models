

\section{Model description}
    In \cite{GatheralRosenbaum2014} the authors considered the following model. 
    Let there be a riskless asset $B_t \equiv 1$, and a risky asset, whose price $S_t$ is defined 
    by the following equations: 
    \begin{align}
        & dS_t          = \alpha S_t dt + \sigma_t S_tdW_t,               \label{model:RFSVasset} \\
        & d\log\sigma_t = \alpha (m - \log\sigma_t) dt + \nu dW_t^H.      \label{model:RFSVvol}
    \end{align}
    The risky asset is being traded in the market in numeraire prices. In our case, $B_t = 1$ RUB for stocks and $1$ GBP for depositary reciepts.
    \begin{definition}
        A model \eqref{model:RFSVasset} -- \eqref{model:RFSVvol} is called a 
        \emph{Fractional Stochastic Volatility Model} (FSV). For a special case $H < 0.5$ the 
        model is called a \emph{Rough Fractional Stochastic Volatility Model} (RFSV) to emphasise a 
        so-called roughness of the trajectories of the fBm. 
        As a stylized fact we shall demand the stationarity off log-increments.
    \end{definition}

    In \cite{Cheridito2003} an exact formula for the autocovariance function of the log-volatility in the RFSV model was derived:
    \begin{multline}\label{formula:CovLog}
        \cov\left[\log \sigma_{t}, \log \sigma_{t+\Delta}\right] =\\= \frac{H(2H-1)\nu^2}{2\alpha^{2H}}\left(e^{-\alpha\Delta} \Gamma(2H-1) + e^{-\alpha\Delta} \int_{0}^{\alpha\Delta}\frac{e^u}{u^{2-2H}}du + e^{\alpha\Delta} \int_{\alpha\Delta}^{\infty}\frac{e^u}{u^{2-2H}}du\right).
    \end{multline}
    Let $m(q, \Delta, \pi^n)$ be a sample $q$-th absolute moment of $\log RV_{t+\Delta} - \log RV_t$:
    \begin{equation}
        m(q, \Delta, \pi^n) := \frac{1}{n} \sum_{t} \left|\log RV_{t + \Delta} - \log RV_t\right|^q,
    \end{equation}
    i.e. $m(q, \Delta, \pi^n)$ is an empirical counterpart of $\mathbb{E}\left[\left|\log RV_{\Delta} - \log RV_0\right|^q\right]$.
    In this work we shall use the uniform partition of time scale with each step being equal to $15$ minutes, so we omit the $\pi^n$ notation and use $m(q, \Delta)$.
    Via the explicit formula for the covariance function of the log-volatility in the RFSV model \eqref{formula:CovLog}, we can write 
    a closed-form expression for a theoretical $m(2, \Delta)$:
    \begin{equation}
        m(2, \Delta) = 2\left(\var \log\sigma_{t} - \cov\left[\log\sigma_{t}, \log\sigma_{t+\Delta}\right]\right).
    \end{equation}
